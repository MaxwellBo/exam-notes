\documentclass[landscape]{cheat}
\usepackage{amsmath}
\usepackage{amssymb}
\usepackage{listings}

\begin{document}
\footnotesize
\begin{multicols*}{3}

\begin{center}
\Large{\underline{CSSE3100 Cheat Sheet}} \\
\end{center}

%% START HERE
\section{Predicate Logic}

\subsection{Things to remember}
\begin{itemize}
    \item Array indexes must be within the bounds of an array
    \item $\forall$ statements need to specify the set of values
    \item Check edge cases of your functions
\end{itemize}

\subsection{Example Function}
Specify a function $all\_elems(a, i, j, v)$ which is true when all elements of array $a$ from position $i$ up to (but not including) position $j$ have the value $v$.
\begin{multline*}
all\_elems(a, i, j, v) =
    0 \leq i
    \land i \leq j
    \land j \leq a.len \\
    \land (\forall k : \mathbb{N} \cdot
        i \leq k
        \land k < k
        \Rightarrow a[k] = v
    )
\end{multline*}

\subsection{Example Specification}
Specify a program which solves the Dutch National Flag problem.
This problem requires sorting an array of red, white and blue elements so that the red elements come first, then the white elements and then the blue.
\begin{multline*}
\texttt{requires } a = a0 \land only\_elems(a, red, white, blue) \\
\end{multline*}

\begin{multline*}
\texttt{ensures } permutation(a, a0) \land \\
    \exists w, b : \mathbb{Z} \cdot 0 \leq w \land w \leq b \land b \leq a.len \\
    \land all\_elems(a, 0, w, red) \\
    \land all\_elems(a, w, b, white) \\
    \land all\_elems(a, b, a.len, blue)
\end{multline*}

\section{Hoare Logic}

\subsection{Loop Invariants $I$}
\begin{itemize}
    \item Used for partial correctness and total correctness
    \item True at the start of the loop: $P \Rightarrow U(I)$
    \item Maintained during the loop: $\{I \land b\}[]S_1\{I\}$
    \item At the end of the loop, must be equivalent to postcondition: $\{I \land \lnot b \}[]S\{Q\}$
\end{itemize}
Strategies:
\begin{itemize}
    \item Replace constant with variable that approaches constant in postcondition 
    \item Factor out $\lnot b$ from postcondition
    \item Place bounds on the range of all values that change during the loop
\end{itemize}

\subsection{Loop Variants $v$}
\begin{itemize}
    \item Used for total correctness
    \item Non-negative at the start of the loop: $P \Rightarrow U(v \geq 0)$
    \item Always non-negative: $0 \leq v$
    \item Always decreasing $v < v_0$
\end{itemize}
Strategies:
\begin{enumerate}
    \item Find variables that change during your loop, and adjust signs to make them decrease
    \item Combine decreasing variables in a way that the net sum is always decreasing
    \item Add a constant so that $v = 0 \Rightarrow b$
\end{enumerate}

\section{JML}
\subsection{Things to Remember}
\begin{itemize}
    \item Everything is non-nullable by default
    \item Invariants are implicitly in the pre- and post-conditions of all methods that don't specify \texttt{/*@ helper @*/}
    \item Mark pure things as \texttt{/*@ pure @*/} where appropriate
    \item Array ranges \texttt{a[i..j]} are inclusive
    \item Add \texttt{/*@ spec\_public @*/} where appropriate
\end{itemize}

\subsection{Example Snippets}
Array that can contain null
\begin{lstlisting}
    public /*@ nullable @*/ Object[] h;
    /*@ invariant h != null; @*/
\end{lstlisting}

Function that can throw exception
\begin{lstlisting}
    /*@ public normal_behaviour
      @ requires ...
      @ ensures ...
      @ assignable ...
      @
      @ also
      @
      @ public exceptional_behaviour
      @ requires ...
      @ signals_only HashTableFullException;
      @ signals (HashTableFullException e) true;
      @ assignable \nothing;
      @*/
    public void add(Object o) { ... }
\end{lstlisting}

Array that must be of a specific type
\begin{lstlisting}
    A[] arr = new A[10];
    //@ invariant \typeof(arr) == \type(A[]);
\end{lstlisting}

Recursive function that is guaranteed to terminate
\begin{lstlisting}
    /*@ requires 0 <= x && 0 <= y;
      @ ensures \result = x * y;
      @ measured_by x;
      @*/
    public int mult(int x, int y) {
        if (x == 0) {
            return 0;
        } else {
            return y + mult(x - 1, y);
        }
    }
\end{lstlisting}

\section{Weakest Preconditions}

\section{Refinement}

\end{multicols*}
\end{document}
