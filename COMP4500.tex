\documentclass[a4paper,landscape]{article}
\usepackage{amsmath}
\usepackage{listings}

\usepackage{mathptmx}
\usepackage[margin=0.3in]{geometry}
\usepackage{multicol} % Imports the column
\setlength{\columnseprule}{0.2pt} % Divide size, was 0.4pt
\setcounter{secnumdepth}{0}
\setlength{\parindent}{0pt}
\setlength{\parskip}{0pt}
\pagestyle{empty}
\usepackage[compact]{titlesec}
\titlespacing{\section}{0pt}{2ex}{1ex}
\titlespacing{\subsection}{0pt}{1ex}{0ex}
\titlespacing{\subsubsection}{0pt}{0.5ex}{0ex}
\linespread{0.9}

\usepackage{paralist}


\begin{document}
\begin{multicols*}{4}

\begin{align*}
    f \in O(g) \Leftrightarrow& \forall n > n_0, 0 \leq f(n) \leq c \cdot g(n) \\
    f \in \Theta(g) \Leftrightarrow& \forall n > n_0, 0 \leq c_1 \cdot g(n) \leq f(n) \leq c_2 \cdot g(n) \\
    f \in \Omega(g) \Leftrightarrow& \forall n > n_0, 0 \leq c \cdot g(n) \leq f(n) \\
    f \in o(g) \Rightarrow& \lim_{n \rightarrow \infty} \frac {f(n)} {g(n)} = 0 \\
    f \in \omega(g) \Rightarrow& \lim_{n \rightarrow \infty} \frac {g(n)} {f(n)} = 0 \\
% TODO: asymp: 22 and 23
\end{align*}
\begin{align*}
    \sum_{k=1}^{n} k =& \frac{n(n + 1)}{2}\\
    \sum_{k=1}^{n} k^2 =& \frac{n(n + 1)(2n + 1)}{6}\\
    \sum_{k=1}^{n} k^3 =& \frac{n^2 (n + 1)^2}{4}\\
    \sum_{k=0}^{n} x^k =& \frac{x^{n+1} - 1}{x - 1} \text{ for } x \neq 1\\
    \sum_{k=0}^{\infty} x^k =& \frac{1}{1 - x} \text{ for } |x| < 1\\
    \sum_{k=1}^{n} \frac{1}{k} =& \ln n + O(1) (\rightarrow \gamma ~ 0.577 \ldots)\\
    \sum_{i=1}^{n} 1^k \approx& \frac{n^{k +1}}{|k + 1|} \text{ if } k \neq -1\\
    \lim_{n \rightarrow a} \frac{f(n)}{g(n)} =& \lim_{n \rightarrow a} \frac{f'(n)}{g'(n)}\\
    \sum_{k=1}^{n} (a_k - a_{k - 1}) =& a_n - a_0\\
    \sum_{k=0}^{n-1} (a_k- a_{k + 1}) =& a_0 - a_n\\
    \sum_{k=0}^{\infty} x^k =& \frac{1}{1 - x}\\
    \sum_{k=0}^{\infty} kx^k =& \frac{x}{(1 - x)^2}\\
\end{align*}
\begin{align*}
    \lg \prod_{k = 1}^{n} a_k =& \sum_{k = 1}^{n} \lg a_k\\
    b^{\log_b y} =& y\\
    x^a = b \Leftrightarrow & \log_x b = a\\
    a^{\log_b n} =& n^{log_b a}\\
    \log_a b =& \frac{\log_c b}{\log_c a} \text{ for } c > 0\\
    \log(ab) =& \log a + \log b\\
    \log(\frac{1}{a}) =& -\log a\\
    \log(a^b) =& b \log a\\
    (\lg n)^{\lg n} =& n^{\lg \lg n}\\
    n^{\frac{1}{\lg n}} =& 2\\
    2^{\sqrt{2 \lg n}} =& n^{\sqrt{\frac{2}{\lg n}}}\\
    (\sqrt{2})^{\lg n} =& \sqrt{n}\\
% TODO: math: 16 and 17
\end{align*}

\subsubsection{Recurrences}
\begin{align*}
    T(n) =& aT(\frac{n}{b}) + D(n) + C(n)
\end{align*}

where $D$ is the time to divide the problem, and $C$ is the time to combine subsolutions.

% TODO: rec: 11

\begin{align*}
    T(n) =& 2T(\lfloor \sqrt{n} \rfloor) + \lg n\\
    T(2^m) =& 2T(2^{m/2}) + m \text{ where } m = \lg m\\
    S(2m) =& 2S(m / 2) + m \text{ where } S(m) = T(2^m)\\
    T(n) =& T(2^m) = S(m) = \Theta(m \lg m)\\
        =& \Theta(\lg n \lg \lg n)
\end{align*}


\subsubsection{Master Theorem}
\begin{description}
    \item[Find] $a \geq 1$ and $b \geq 1$ such that $T(n) = a T\left(\frac n b\right) + f(n)$
\end{description}
\begin{align*}
    T(n) = a T\left(\frac n b\right) + f(n) \\
    \frac{f(n)}{n^{\log_b a}} \in O(n^{-\epsilon})  \Rightarrow& T(n) \in \Theta(n^{\log_b a}) \text { for some } \epsilon > 0  \\
    \frac{f(n)}{n^{\log_b a}} \in \Theta(1) \Rightarrow& T(n) \in \Theta(n^{\log_b a} \lg n) \\
    \frac{f(n)}{n^{\log_b a}} \in \Omega(n^{\epsilon}) \Rightarrow& T(n) \in \Theta(f(n)) \text{ for some } \epsilon > 0 \\&\text{ and } af(\frac{n}{b}) \leq cf(n) \text{ for some } c < 1 \\
    f(n) \in \Theta(n^{\log_b a} \lg^{k} n) \Rightarrow& T(n) \in \Theta(n^{\log_b a} \lg^{k + 1} n)\\ &\text{if } f(n) \text{ is larger}\\ &\text{but not polynomial larger than } n^{\log_b a}\\
\end{align*}

\begin{description}
    \item[Polynomially Larger Than]
\end{description}
\begin{align*}
    f(n) > g(n) \Leftrightarrow& \exists \epsilon. \epsilon > 0 \land f(n) \in \Omega(g(n) \cdot n^\epsilon)
\end{align*}

\begin{description}
    \item[Regularity] $f$ in Case 3 must be regular
\end{description}
\begin{align*}
    \exists c. c < 1 \land af\left(\frac n b\right) < cf(n)\\
\end{align*}

\subsubsection{Accounting method}
\begin{compactitem}
    \item Recall actual cost
    \item Define amoritised cost
    \item Ensure that the amortised cost is an upper bound on the actual cost
\end{compactitem}

The amortised cost must be an upper bound on the actual cost, such that the total amortised cost minus the total actual cost is never negative.

\subsubsection{Potential method}
\begin{compactitem}
    \item Recall actual cost
    \item Define potential function, and ensure that $\forall i. \Phi(D_i) \geq \Phi(D_0)$ for all $i > 0$.
    \item Determine the amoritised cost of each operation from $\Phi$.
\end{compactitem}

\begin{align*}
    \hat{c}_i &= c_i + (\Phi(D_i) - \Phi(D_{i - 1}))
\end{align*}

\begin{compactitem}
    \item A path is a sequence of vertices with edges between them
    \item $u$ is reachable from $v$ if there is a path from $v$ to $u$
    \item A path is:
    \item \begin{compactitem}
        \item simple if all verticies in the path are distinct
        \item a cycle if the start and end vertices are the same, if there are more than 2 vertices
    \end{compactitem}
    \item A cycle is simple if all the vertices and edges are distinct
    \item An undirected graph is :
        \begin{compactitem}
            \item connected if every vertex is reachable from all other vertices
            \item a forest if it is acyclic
            \item a tree if it is a forest with only one connected component
        \end{compactitem}
    \item An directed graph is strongly connected if every two vertices are reachable from each other
    \item $G' = (V', E')$ is a sub-graph of $G = (V, E)$ when $V' \subseteq V$ and $E' \subseteq E$
    \item $G'$ is a spanning sub-graph of G if $V' = V$
    \item The sub-graph of $G = (V, E)$ that is induced by $V'$ is the $G' = (V', E')$ where $E' = \{(u, v) \in E: u \in V' \wedge v \in V' \}$
    \item Adjacency lists are efficient for spare graphs, where there are few edges relative to the number of vertices, as opposed to adjacency matrices
\end{compactitem}

\texttt{isAdjacentTo(v, u)}, using list representation is $\Theta(V)$, and using matrix representation it's $\Theta(1)$.

\texttt{neighbours(v)}, using list adjacency is $\Theta(V + E)$, and using matrix representation it's $\Theta(V^2)$

\subsubsection{Handshake Lemma}
\begin{lstlisting}
for v in G.V
    for e in G.Adj(e)
        OPERATION
\end{lstlisting}
\begin{description}
    \item[Complexity] $\Theta((V+E) \times OPERATION)$ for adjacency list.
    \item[Complexity] $\Theta((V+E) \times OPERATION + V^2)$ for adjacency matrix.
\end{description}

\subsubsection{Breadth First Search}
\begin{description}
    \item[Complexity] $\Theta(V+E)$ for adjacency list.
    \item[Complexity] $\Theta(V^2)$ for adjacency matrix.
    \item[Finds] breadth first tree of graph.
    \item[Finds] shortest path to each node from source, assuming equal weights.
\end{description}

% TODO: L3-2018: 46 back and forward edges
\begin{lstlisting}
DFS(G, s)
    time = 0
    for u in G.V - {S}
        u.colour = white
        u.parent = NIL
    s.colour = grey
    s.parent = NIL
    for u in G.V
        if u.colour == white
            DFS-VISIT(G, u)

DFS-VISIT(G, u)
    time = time + 1
    u.discovery = time
    for v in G.Adj(u)
        if v.colour == white
            v.colour = grey
            v.parent = u
            DFS-VISIT(G, v)
    u.colour = black
    time = time + 1
    u.finish = time
\end{lstlisting}
\begin{description}
    \item[Complexity] $\Theta(V+E)$ for adjacency list.
    \item[Complexity] $\Theta(V^2)$ for adjacency matrix.
    \item[Finds] depth first forests of graph.
    \item[Finds] loops, if you find a grey vertex in DFS-VISIT.
    \item[Finds] a topological sort, if you output in descending order by $u.finish$.
\end{description}

\begin{lstlisting}
STRONGLY-CONNECTED-COMPONENTS(G)
\end{lstlisting}
\begin{compactitem}
    \item call $DFS(G)$
    \item compute $G^T$
    \item call $DFS(G^T)$ but consider vertices in
        decreasing order of $u.f$
    \item return discrete trees of second search
\end{compactitem}

\begin{lstlisting}
TOPOLOGICAL-SORT(G)
\end{lstlisting}
\begin{compactitem}
    \item initialise empty linked-list of verts
    \item call $DFS(G)$
    \item as each vert is finished, insert onto front of a linked list
    \item return the list of vertices
\end{compactitem}

\begin{lstlisting}
MST-KRUSKAL(G, w)
    A = {}
    for u in G.V
        MAKE-SET(u)
    SORT(G.E, w)
    for (u, v) in G.E
        if FIND-SET(u) != FIND-SET(v)
            A = A + {(u, v)}
            UNION(u, v)
    return A
FIND-SET(x)
if x != x.p
    x.p = FIND-SET(x.p) 
    // ^ performs path compression
    return x.p
\end{lstlisting}
\begin{description}
    \item[Finds] Minimum spanning tree - subset of edges such that all vertices are connected and sum of edge weight is minimised.
    \item[Complexity] $\Theta(E\lg{E})$ using union-find trees
    \item[Notes] \texttt{UNION} ensures the tree with fewer nodes is made to point ot the tree with more nodes (to avoid updating \texttt{rank}) (unless they have equal rank, then you have to increment rank of the new representative)
\end{description}

\begin{lstlisting}
PRIORITY-FIRST-SEARCH(G, s, ...)
    INIT-SINGLE-SOURCE(G, s)
    Q = G.V
    while Q != {}
        u = EXTRACT-MIN(Q)
        for v in G.Adj(u)
            RELAX(u, v, ...)

RELAX(u, v, w)
    if v.d > PRIORITY
        v.d = PRIORITY
        v.pi = u
\end{lstlisting}
\begin{description}
    \item[Complexity] $\Theta(V)\cdot T_{EXTRACT} + \Theta(E)\cdot T_{DECREASE}$
    \item[Complexity] $O(E\lg{V})$ for binary heap.
    \item[Complexity] $O(E + V\lg{V})$ for fibonacci heap.
\end{description}

\subsubsection{Prim's Minimum Spanning Tree}
Priority first search with $PRIORITY(u, v, w) = w(u, v)$.
\begin{description}
    \item[Complexity] Same as priority first search.
    \item[Finds] Minimum spanning tree.
\end{description}

\subsubsection{Dijkstra'a Algorithm}
Priority first search with $PRIORITY(u, v, w) = u.key + w(u, v)$.
\begin{description}
    \item[Complexity] Same as priority first search.
    \item[Finds] Single source shortest path to any node, assuming no negative edges.
\end{description}

\begin{lstlisting}
BELLMAN-FORD(G, w, s)
    INIT-SINGLE-SOURCE(G, s)
    for i in 1..|G.V|-1
        for u,v in G.E
            RELAX(u, v, w)
    for u,v in G.E
        if v.d > u.d + w(u,v)
            return FALSE
    return TRUE
\end{lstlisting}

\begin{description}
    \item[Complexity] $O(V\cdot E)$
    \item[Finds] Single source shortest path to any node, assuming no negative cycles.
    \item[Finds] The existence of negative cycles.
    \item[Note] For every vertex $v$, $v.d$ has reached its final value if the number of iterations $\geq$ the length of any shortest weight path to $v$.7
\end{description}

\begin{description}
    \item[P] problems are those that can be solved in polynomial time by a deterministic turing machine.
    \item[NP] problems are those that can be solved in polynomial time by a nondeterministic turing machine.
        Also the problems that a solution (certificate) to which can be \emph{verified} in polynomial time by a deterministic turing machine.
    \item[NP hard] problems are problems that any problem in NP can be reduced to in polynomial time.
    \item[NP complete] problems are NP hard problems that are in NP.
    \item[P = NP] probably not.
\end{description}

\begin{align*}
L \in NP \wedge L \in P &\Rightarrow P = NP\\
L \in NPC &\Leftrightarrow L \in NPH \wedge L \in NP\\
\end{align*}

Say we can reduce an instance $\alpha$ of $A$ to an instance $\beta$ of $B$ in polynomial time. If $B$ is easy, then so is $A$. If $A$ is hard, then so is $B$.

Let $X$ be your own problem. If you can show that CIRCUIT-SAT is polynomial-time reducible to $X$, then $X$ is $NP$-hard. To show that $X$ is $NP$-complete you also need to show it is in $NP$.


CIRCUIT-SAT, SAT, 3-CNF-SAT, CLIQUE (SUBSET-SUM), VERTEX-COVER, HAM-CYCLE, TSP


\end{multicols*}
\end{document}
